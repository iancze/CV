\documentclass[10pt]{article}
\usepackage{geometry}
\usepackage[T1]{fontenc}
\usepackage{url}

\usepackage{lmodern}

%\usepackage[hidelinks]{hyperref}

\usepackage[dvipsnames]{xcolor}
\definecolor{myblue1}{RGB}{22, 63, 130}
\definecolor{mygrey1}{RGB}{145, 145, 145}

% \usepackage[backref,breaklinks,colorlinks,urlcolor=blue,citecolor=blue,linkcolor=blue]{hyperref}
\usepackage[backref,breaklinks,colorlinks,urlcolor=myblue1,citecolor=myblue1,linkcolor=myblue1]{hyperref}
\usepackage{etaremune}

\usepackage[thinlines]{easytable}

\usepackage{aas_macros}

\usepackage{multibbl}

\newbibliography{all}
\newbibliography{first}
\newbibliography{preprint}

\pagestyle{empty}
\geometry{letterpaper,tmargin=0.7in,bmargin=0.7in,lmargin=0.7in,rmargin=0.7in,headheight=0in,headsep=0in,footskip=.3in}


% Change spacing of the itemize and etaremune environments
\setlength{\parindent}{0in}
\setlength{\parskip}{1em}
\setlength{\itemsep}{1em}
\setlength{\topsep}{0in}
\setlength{\tabcolsep}{0in}

% Increase the spacing between rows in the tabular environment
\renewcommand{\arraystretch}{1.05}

\newcommand{\hr}{\rule{\textwidth}{0.2pt}}
\newcommand{\rowskip}{1.2mm}

\makeatletter
\renewcommand{\section}{\@startsection{section}{1}{0pt}{-\baselineskip}{0.5\baselineskip}{\scshape\color{myblue1}}}
\renewcommand{\subsection}{\@startsection{subsection}{2}{0pt}{-\baselineskip}{0.5\baselineskip}{\hspace{12pt}\itshape}}

\makeatother

\begin{document}

\fontfamily{ppl} \selectfont

{\huge \textcolor{myblue1}{Dr. Ian Czekala} } {\small \hfill Updated \today}

\rule{\textwidth}{1pt}

\parbox[t]{3in}{
\begin{flushleft}
Campbell Hall Rm. 605B\\
University of California at Berkeley\\
Berkeley, CA 94720-3411\\
ORCID ID: \href{http://orcid.org/0000-0002-1483-8811}{0000-0002-1483-8811} \\
\end{flushleft}} \ \hfill  \
\parbox[t]{3in}{
\begin{flushright}
%Office: Rm 216 Physics and Astronomy Building\\
Phone: (631)-793-9292\\
Email: iczekala@berkeley.edu\\
\url{http://iancze.github.io}\\
U.S. Citizen
\end{flushright}}

\vspace{12pt}

\section*{Scientific Interests}
Protoplanetary disks, exoplanets, star and planet formation, astrostatistics, radio interferometry, spectroscopy

% Try either https://www.sharelatex.com/learn/Tables#Changing_the_appearance_of_a_table
% or easytable:

\section*{Education}
\begin{tabular*}{\textwidth}{@{\hspace{10pt}}p{1in}l}
  2012 - 2016 & \emph{Ph.D.} in Astrophysics, Harvard University, Cambridge, MA \\
              & advisor Sean M. Andrews\\
  2010 - 2012 & \emph{Masters of Arts} in Astronomy and Astrophysics, Harvard University \\
              & advisor Edo Berger\\
  2006 - 2010 & \emph{Bachelor of Science}, Aerospace Engineering, Astronomy\\
  & Graduation with High Distinction, University of Virginia (UVA), Charlottesville, VA\\
\end{tabular*}

\section*{Scientific Research}
\begin{tabular*}{\textwidth}{@{\hspace{10pt}}p{1in}l}
2018 - present & \emph{Architectures and Dynamics of Protoplanetary Systems}, Postdoctoral Advisor Eugene Chiang \\
2016 - 2018 & \emph{Disk and Stellar Dynamics of Pre-Main Sequence Systems}, Postdoctoral Advisor Bruce Macintosh \\
2013 - 2016 & \textbf{Ph.D. Thesis}: \emph{The Fundamental Properties of Young Stars}, CfA, advised by Sean Andrews\\
2012 & \emph{MMTCam Commissioning}, Harvard-Smithsonian CfA, advised by Warren Brown\\
2010 - 2012 & \textbf{Masters project}: \emph{Intermediate Luminosity Transients}, Harvard University, advised by Edo Berger\\
2009 - 2010 & \emph{PAPER Instrumentation Study}, University of Virginia, advised by Richard Bradley\\
2009 - 2010 & \emph{ALMA Collaborative Engineering Study}, Santiago, Chile, advised by Kelsey Johnson and Alison Peck\\
2009 & \emph{Circumstellar Disks}, Smithsonian Astrophysical Observatory REU Intern, advised by Dr.~Sean~Andrews\\
\end{tabular*}

\section*{Professional Appointments}
\begin{tabular*}{\textwidth}{@{\hspace{10pt}}p{1in}l}
2018 - present & NASA Hubble Fellowship Program (NHFP) Sagan Postdoctoral Fellow\\
 & University of California Berkeley\\
2016 - 2018 & Porat Postdoctoral Fellow\\
 & Kavli Institute for Particle Astrophysics and Cosmology, Stanford University \\
2010 - 2016 & Graduate Student \\
 & Harvard University\\
\end{tabular*}

\section*{Honors and Awards}
\begin{tabular*}{\textwidth}{@{\hspace{10pt}}p{1in}l}
2013, 2014 & \emph{(2) Certificates of Distinction in Teaching}, Harvard University \\
2011 - 2016 & NSF \emph{Graduate Research Fellowship}\\
2006 - 2010 & \emph{Jefferson Scholar}, UVA, full scholarship\\
2006 - 2010 & \emph{Rodman Scholar}, UVA\\
2010 & \emph{Outstanding SEAS Student}, UVA\\
2010 & \emph{Louis T. Rader} Award for Mechanical and Aerospace Engineering\\
& School of Engineering and Applied Sciences, UVA\\
2010 & \emph{21 Society Fourth Year Recognition}, UVA\\
2010 & \emph{Limber Award}, UVA Astronomy Department\\
\end{tabular*}

\section*{Refereed Publications}

First author: 7 / total: 31 / citations (all): 1610 / h-index (all): 20 / (2019-08-01) [\href{https://ui.adsabs.harvard.edu/#/public-libraries/G0Ow9TGTRyuVT7hbhzailA}{link}]


% cite everything in library silently, creating only refs section
\nocite{first}{*}

\nocite{all}{*}

\nocite{preprint}{*}

\bibliographystyle{first}{cv}
\bibliography{first}{first_papers}{First and Second Author Publications}

\bibliographystyle{all}{cv}
\bibliography{all}{all_papers}{Many-Author Publications}

\bibliographystyle{preprint}{cv}
\bibliography{preprint}{preprint_papers}{Preprints and Non-Refereed Papers}

\section*{Students Advised}
\begin{itemize}
  \item Mr. Joseph Michael Akana Murphy, Stanford University Coterminal Masters Student \\
  Summer Research and Senior Thesis; 2017 - 2019\\
  \emph{Unveiling the Spectra of Young Stars with Gaussian Processes: Applications to LkCa~15}
\end{itemize}

\section*{Invited Research Talks and Presentations}
\begin{tabular*}{\textwidth}{@{\hspace{10pt}}p{1.2in}l}
  % (upcoming) & Frank Bash Symposium, UT Austin \\
  Mar 14, 2019 & Department lunch talk, UC Berkeley, CA \\
  & \emph{Circumbinary Planets and Disks} \\[\rowskip]
  Feb 6, 2019 & SOFIA colloquium, NASA Ames, Mountain View, CA \\
  & \emph{The Degree of Alignment of Circumbinary Disks and their Host Binaries} \\[\rowskip]
  Nov 29, 2018 & Weekly seminar, Columbia University, NYC, NY \\
  & \emph{The Alignment of Binary Star Orbits and their Circumbinary Disks} \\[\rowskip]
  Nov 28, 2018 & Stars Meeting, Flatiron Institute, NYC, NY \\
  & \emph{The Alignment of Binary Star Orbits and their Circumbinary Disks} \\[\rowskip]
  Nov 8, 2018 & Sagan Fellows Symposium at Caltech, Pasadena, CA \\
  & \emph{The Alignment of Binary Star Orbits and their Circumbinary Disks} \\[\rowskip]
  Nov 7, 2018 & CIPS Planet and Star Formation Seminar, UC Berkeley, CA  \\
  & \emph{The Alignment of Binary Star Orbits and their Circumbinary Disks} \\[\rowskip]
  Apr 24, 2018 & KIPAC Tea Talk at Stanford University, Palo Alto, CA \\
  & \emph{Using Gaussian Processes to Construct Flexible Models of Stellar Spectra} \\[\rowskip]
  Jan 10, 2018 & AAS Special Session on Gaussian Processes and Machine Learning, Washington, D.C. \\
  & \emph{Using Gaussian Processes to Construct Flexible Models of Stellar Spectra}\\[\rowskip]
  Oct 18, 2017 & CIPS Planet and Star Formation Seminar, UC Berkeley, CA \\
  & \emph{Protoplanetary Disks around Pre-Main Sequence Binary Stars} \\[\rowskip]
  June 1, 2017 & NAOJ Star and Planet Formation Seminar, NAOJ, Tokyo, Japan \\
  & \emph{Protoplanetary Disks around Pre-Main Sequence Binary Stars} \\[\rowskip]
  May 31, 2017 & RIKEN Star and Planet Formation Seminar, RIKEN, Tokyo, Japan \\
  & \emph{Protoplanetary Disks around Pre-Main Sequence Binary Stars} \\[\rowskip]
\end{tabular*}
\begin{tabular*}{\textwidth}{@{\hspace{10pt}}p{1.2in}l}
  May 25, 2017 & Kavli Institute for Astronomy and Astrophysics Colloquium, Peking University, Beijing, China \\
  & \emph{Protoplanetary Disks around Pre-Main Sequence Binary Stars} \\[\rowskip]
  May 16, 2017 & Harvard Astrostatistics Seminar, Harvard University, Cambridge, MA \\
  & \emph{Disentangling Spectra With Gaussian Processes: Applications to Radial Velocity Analysis} \\[\rowskip]
  Aug 23, 2016 & SAMSI Astrostatistics Opening Workshop, Research Triangle Park, NC \\
  & \emph{Systematics-Dominated Spectroscopic Inference} \\[\rowskip]
  Jul 20, 2016 & ASIAA Colloquium, Taipei, Taiwan \\
  & \emph{The Fundamental Properties of Young Stars} \\[\rowskip]
  Jul 5, 2016 & ASIAA Star Formation Meeting, Taipei, Taiwan \\
  & \emph{Disk-Based Dynamical Masses and Applications with the SMA} \\[\rowskip]
  Jun 9, 2016 & Kavli Institute for Astronomy and Astrophysics Lunch Seminar, Peking University, Beijing, China \\
  & \emph{The Fundamental Properties of Young Stars} \\[\rowskip]
  Mar 8, 2016 & CfA Exoplanet Lunch, Harvard-Smithsonian Center for Astrophysics \\
  & \emph{Using Protoplanetary Disks to Precisely Weigh Stars} \\[\rowskip]
  Feb 9, 2016 & BU Lunch Talk, Boston University, Boston, MA \\
  & \emph{Using Protoplanetary Disks to Weigh the Youngest Stars and} \\
  & \emph{Constrain The Earliest Stages of Stellar Evolution} \\[\rowskip]
  Dec 10-11, 2015 & ISM Seminar at UT Austin, Austin, TX \\
  & \emph{Using Protoplanetary Disks to Weigh the Youngest Stars and} \\
  & \emph{Constrain The Earliest Stages of Stellar Evolution} \\[\rowskip]
  Dec 7-8, 2015 & Tea Talk at Caltech, Pasadena, CA \\
  & \emph{Using Protoplanetary Disks to Weigh the Youngest Stars and} \\
  & \emph{Constrain The Earliest Stages of Stellar Evolution} \\[\rowskip]
  Nov 17, 2015 & KIPAC Tea Talk at Stanford University, Palo Alto, CA \\
  & \emph{Using Protoplanetary Disks to Weigh the Youngest Stars and} \\
  & \emph{Constrain The Earliest Stages of Stellar Evolution} \\[\rowskip]
  Nov 16, 2015 & ACES talk at NASA Ames, Mountain View, CA \\
  & \emph{Using Protoplanetary Disks to Weigh the Youngest Stars and} \\
  & \emph{Constrain The Earliest Stages of Stellar Evolution} \\[\rowskip]
\end{tabular*}
\begin{tabular*}{\textwidth}{@{\hspace{10pt}}p{1.2in}l}
  Nov 12-13, 2015 & FLASH talk at UC Santa Cruz, Santa Cruz, CA \\
  & \emph{Using Protoplanetary Disks to Weigh the Youngest Stars and} \\
  & \emph{Constrain The Earliest Stages of Stellar Evolution} \\[\rowskip]
  Nov 4, 2015 & CIPS Planet and Star Formation Seminar, UC Berkeley, CA\\
  & \emph{Using Protoplanetary Disks to Weigh the Youngest Stars and} \\
  & \emph{Constrain The Earliest Stages of Stellar Evolution} \\[\rowskip]
  Apr 22, 2015 & CIPS Planet and Star Formation Seminar, UC Berkeley, CA\\
  & \emph{Flexible Spectroscopic Inference for Young Stars} \\[\rowskip]
  Apr 14, 2015 & Astrostatistics Seminar, Statistics Department, Harvard University, MA \\
  & \emph{Flexible Spectroscopic Inference} \\[\rowskip]
\end{tabular*}

\section*{Contributed Research Talks and Presentations}
\begin{tabular*}{\textwidth}{@{\hspace{10pt}}p{1.2in}l}
  Aug 19-23, 2019 & Extreme Solar Systems IV, Reykjavik, Iceland\\ 
  & \emph{The Mutual Inclinations of the Proto-Tatooine Disks} \\[\rowskip]
  Jul 21-26, 2019 & Great Barriers in Planet Formation conference, Palm Cove, Australia\\
  & \emph{The Degree of Alignment between Circumbinary Disks and their Host Binaries} \\[\rowskip]
  Jun 28, 2019 & Bay Area Exoplanet Meeting, NASA Ames, Mountain View, CA\\
  & \emph{Gradient-based Inference Algorithms for Exoplanet Science} \\[\rowskip]
  Dec 14, 2018 & Bay Area Exoplanet Meeting, NASA Ames, Mountain View, CA\\
  & \emph{The Degree of Alignment between Circumbinary Disks and their Host Binaries} \\[\rowskip]
  Nov 19-23, 2018 & Lorentz Center, Leiden, Netherlands\\
  & \emph{Weighing Stars from Birth to Death} Workshop Presentation \\[\rowskip]
  Jan 9, 2018 & AAS meeting, Washington, D.C. \\
  & \emph{Mutual Inclinations of Circumbinary Protoplanetary Disks} \\[\rowskip]
  Dec 13, 2017 & Exoplanets and Planet Formation, Shanghai, China \\
  & \emph{Mutual Inclinations of Circumbinary Protoplanetary Disks} \\[\rowskip]
  Dec 1, 2017 & Bay Area Exoplanet Meeting, NASA Ames, Mountain View, CA\\
  & \emph{Mutual Inclinations of Circumbinary Protoplanetary Disks} \\[\rowskip]
  Aug 22, 2017 & Exoclipse Conference, Boise State University, Boise, ID\\
  & \emph{Disentangling Stellar Spectra with Gaussian Processes: Applications to Radial Velocity Analysis} \\[\rowskip]
  Mar 3, 2017 & Bay Area Exoplanet Meeting, NASA Ames, Mountain View, CA\\
  & \emph{Disentangling Stellar Spectra with Gaussian Processes: Applications to Radial Velocity Analysis} \\[\rowskip]
  Oct 17-28, 2016 & SAMSI Exoplanet Workshop, Research Triangle Park, NC\\
  & \emph{Modeling Stellar Spectra with Gaussian Processes} \\[\rowskip]
  Jan 7, 2016 & Dissertation talk, AAS Winter Meeting, Kissimmee, FL \\
  & \emph{Using Protoplanetary Disks to Weigh the Youngest Stars and} \\
  & \emph{Constrain The Earliest Stages of Stellar Evolution} \\[\rowskip]
  Oct 19-21, 2015 & Fitting Stars, CMDs, and Galaxies, Rockport, MA \\
  & \emph{Constructing a Likelihood Function for Spectroscopic Inference}\\[\rowskip]
  Sep 18, 2015 & Bay Area Exoplanet Science Meeting, The SETI Institute, Mountain View, CA \\
  & \emph{Using Protoplanetary Disks to Weigh the Youngest Stars and} \\
  & \emph{Constrain The Earliest Stages of Stellar Evolution} \\[\rowskip]
  May 28-29, 2015 & Emerging Researchers in Exoplanet Science Symposium, The Pennsylvania State University\\
  & \emph{Accessing the Fundamental Properties of Young Stars} \\[\rowskip]
  Jun 18-21, 2014 & ExoStat 2014, Carnegie Mellon University, PA \\
  & \emph{Fitting Stellar Spectra With Some Help From Gaussian Processes} \\[\rowskip]
  Apr 27, 2012 & CfA OIR Symposium, Cambridge, MA \\
  & \emph{The Unusually Luminous Extragalactic Nova SN 2010U}\\
% \end{tabular*}
% \begin{tabular*}{\textwidth}{@{\hspace{10pt}}p{1.2in}l}
  Jan 21 - 27, 2012 & Physics of Astronomical Transients, Aspen Center for Physics, Aspen, CO \\
  & \emph{Supernovae Impostors and Pan-STARRS} \\[\rowskip]
  Jun 28 - 30, 2011 & Intermediate Luminosity Red Transients, Space Telescope Science Institute, Baltimore, MD\\
  & \emph{The Unusually Luminous Extragalactic Nova SN 2010U} \\[\rowskip]
  Apr 16, 2010 & ACC Meeting of the Minds Conference, Georgia Institute of Technology \\
  & \emph{Precision Array to Probe the Epoch of Reionization (PAPER) Instrumentation Study} \\[\rowskip]
  Apr 9 - 10, 2010 &  AIAA Region I-MA Student Conference, Virginia Institute of Technology\\
  & \emph{Precision Array to Probe the Epoch of Reionization (PAPER) Instrumentation Study} \\[\rowskip]
\end{tabular*}

\section*{Successful P.I. Proposals}
\begin{tabular*}{\textwidth}{@{\hspace{10pt}}p{1.2in}l}
  Aug 2019 & ALMA Cycle 7: \emph{Mapping the Inner Edge and Interior Cavity of a Kepler-Analog}\\
  & \emph{Circumbinary Protoplanetary Disk}, 4.8 hrs Band 6\\
  Aug 2019 & Automated Planet Finder/Lick : \emph{Identifying Circumbinary Disk Systems with the APF} \\
  & 3 nights \\
  Aug 2019 & Automated Planet Finder/Lick : \emph{Dynamical Masses to Set the Ages of Nearby Young Moving Groups} \\
  & 3 nights \\
  Feb 2019 & Automated Planet Finder/Lick : \emph{Identifying Circumbinary Disk Systems with the APF} \\
  & 4 nights \\
  Feb 2019 & Automated Planet Finder/Lick : \emph{Dynamical Masses to Set the Ages of Nearby Young Moving Groups} \\
  & 3 nights \\
  Aug 2018 & ALMA Cycle 6: \emph{Unlocking the TWA~3 Triple System with ALMA}\\
  & 1.3 hrs Band 6\\
  Aug 2018 & ALMA Cycle 6: \emph{Mapping the Inner Edge of a Kepler-Analog Circumbinary Protoplanetary Disk}\\
  & 5.7 hrs Band 6\\
  Aug 2016 & ALMA Cycle 4: \emph{Resolving the AK Sco Circumbinary Disk}\\
  & 1 hour Band 6\\
  Oct 2014 & CfA Optical and Infrared division: \emph{Pre-Main Sequence Models}\\
  & 1 night on Magellan/MIKE\\
  Jun 2014 & CfA Optical and Infrared division: \emph{Determining the Systematic Error of Veiling}\\
  & 3 nights each on 1.5m/TRES and 1.2m/Keplercam\\
  Oct 2013 & CfA Optical and Infrared division: \emph{Pre-Main Sequence Models} \\
  & 1 night on Magellan/MIKE\\
  Jun 2013 & CfA Optical and Infrared division: \emph{Pre-Main Sequence Models}\\
  &  3 nights each on 1.5m/TRES and 1.2m/Keplercam\\
\end{tabular*}

\section*{Selected Posters}
\begin{etaremune}
\item \href{https://figshare.com/articles/The_Degree_of_Alignment_Between_Circumbinary_Disks_and_Their_Host_Binaries/9209357}{\emph{The Degree of Alignment Between Circumbinary Disks and their Host Binaries}}\\
\textbf{Ian Czekala}, E.~Chiang, S.~M.~Andrews, E.~L.~N.~Jensen, G.~Torres, D.~J.~Wilner, K.~G.~Stassun, \& B. Macintosh \\
New Horizons in Planetary Systems, Victoria, BC, Canada. May 13-17, 2019
\item \href{https://figshare.com/articles/Using_Protoplanetary_Disks_to_Weigh_the_Youngest_Stars_and_Constrain_The_Earliest_Stages_of_Stellar_Evolution/1613531}{\emph{Using Protoplanetary Disks to Weigh the Youngest Stars and Constrain}}\\
\href{https://figshare.com/articles/Using_Protoplanetary_Disks_to_Weigh_the_Youngest_Stars_and_Constrain_The_Earliest_Stages_of_Stellar_Evolution/1613531}{\emph{The Earliest Stages of Stellar Evolution}} \\
\textbf{Ian Czekala}, S.~M.~Andrews, E.~L.~N.~Jensen, K.~G.~Stassun, D.~Latham, D.~J.~Wilner, \& G.~Torres\\
Extreme Solar Systems III Conference, Waikoloa Village, HI, Nov 29 - 4, 2015
\item \emph{A Disk-based Dynamical Mass Estimate for the Young Binary AK Sco}\\
\textbf{Ian Czekala}, S.~M.~Andrews, E.~L.~N.~Jensen, K.~G.~Stassun, G.~Torres, \& D.~J.~Wilner\\
2015 Gordon Research Conference on Origins of Solar Systems, Mount Holyoke, MA
\item \emph{A Novel Tool for the Spectroscopic Inference of Fundamental Stellar Parameters}\\
\textbf{Czekala, Ian};  Andrews, Sean M.; Latham, David W.; Torres, Guillermo\\
Summer AAS Meeting \#224 \#322.01, Boston, MA
\item \emph{The Unusually Luminous Extragalactic Nova SN 2010U}\\
\textbf{Czekala, Ian}; Chornock, R.; Berger, E.; Pastorello, A.; Marion, G. H.; Challis, P.; Wheeler, J. C.; Botticella, M. T.; Smartt, S.; Ergon, M.; Sollerman, J.\\
American Astronomical Society, AAS Meeting \#218, \#127.11; Vol. 43, 2011
\item \emph{Truncated Disks in TW Hya Association Multiple Star Systems}\\
\textbf{Czekala, Ian}; Andrews, Sean\\
American Astronomical Society, AAS Meeting \#215, \#428.05; Vol. 42, p.345 awarded \textbf{Chambliss Student Achievement Award}
\end{etaremune}

\section*{Workshops and Conferences}
\begin{tabular*}{\textwidth}{@{\hspace{10pt}}p{1.4in}l}
Jun 23 - 28, 2013 & \emph{Gordon Research Conference on Origins of Solar Systems}, Mount Holyoke, MA\\
May 29 - Jun 5, 2012 & \emph{NRAO Summer School on Interferometry and Aperture Synthesis}, Socorro, NM\\
Sept 14 - 16, 2011 & \emph{NRAO CASA Reduction Workshop}, Socorro, NM\\
Sept 18 - 21, 2011 & \emph{PAN-STARRS Science Consortium Meeting}, Cambridge, MA\\
Aug 24 - 25, 2011 & \emph{Derek Bok Teaching Conference}, Harvard University, Cambridge, MA\\
Sept 22, 2009 & \emph{The Fourth North American ALMA Science Center Conference}, Charlottesville, VA\\
\end{tabular*}

\section*{Open Source Code Packages}
\begin{tabular*}{\textwidth}{@{\hspace{10pt}}p{1.4in}l}
PSOAP & Disentangling of Stellar Spectra for Radial Velocity Analysis \\
& \url{https://github.com/iancze/PSOAP} \\
& ASCL: \url{http://adsabs.harvard.edu/abs/2017ascl.soft05013C} \\[\rowskip]
DiskJockey & UV plane modeling of sub-mm interferometric protoplanetary disk observations\\
& \url{https://github.com/iancze/DiskJockey} \\
& ASCL: \url{http://adsabs.harvard.edu/abs/2016ascl.soft03011C}\\[\rowskip]
Starfish & Modular tools for spectroscopic inference \\
& \url{http://iancze.github.io/Starfish/}  \\
& ASCL: \url{http://adsabs.harvard.edu/abs/2015ascl.soft05007C}\\
\end{tabular*}


\section*{Observing Experience}
\subsection*{Magellan Clay 6.5 Meter, Las Campanas Observatory, Chile}
\begin{tabular*}{\textwidth}{@{\hspace{20pt}}p{1.2in}l}
Jul 3-4, 2015 & \emph{MIKE} Pre-Main Sequence Models\\
May 22-23, 2014 & \emph{MIKE} Pre-Main Sequence Models\\
Oct 20-21, 2011 & \emph{LDSS-3} and \emph{MagE} GRB host galaxies and supernovae candidates from Pan-STARRS\\
Jan 11-12, 2011 & \emph{LDSS-3} GRB host galaxies and supernovae candidates from Pan-STARRS\\
\end{tabular*}

\subsection*{Multiple Mirror Telescope 6.5 Meter, Fred Lawrence Whipple Observatory, Arizona}
\begin{tabular*}{\textwidth}{@{\hspace{20pt}}p{1.2in}l}
Nov 26-28, 2011 & \textit{BlueChannel} Pan-STARRS supernova and variable stars \\
Feb 21-23, 2011 & \textit{BlueChannel} Pan-STARRS supernova and variable stars \\
\end{tabular*}
\subsection*{Commissioning}
\begin{tabular*}{\textwidth}{@{\hspace{20pt}}p{1.2in}l}
Jun - Aug, 2012 & \emph{MMTCam} commissioning and installation at MMT\\
\end{tabular*}

\subsection*{The Submillimeter Array Interferometer, Mauna Kea, Hawaii}
\begin{tabular*}{\textwidth}{@{\hspace{20pt}}p{1.2in}l}
  Feb 20-24, 2014 & SMA queue observing\\
  Nov 6 - 10, 2014 & SMA queue observing\\
  Jan 14 - 20, 2015 & SMA queue observing\\
\end{tabular*}

\subsection*{Gemini Planet Imager (GPI), Gemini South, Chile}
\begin{tabular*}{\textwidth}{@{\hspace{20pt}}p{1.2in}l}
  Nov 16-18, 2016 & GPI Exoplanet Survey\\
\end{tabular*}


\section*{Teaching}
\begin{tabular*}{\textwidth}{@{\hspace{10pt}}p{1.4in}l}
Jan - May 2013 & \emph{Teaching Fellow}, AY 193: Noise and Data Analysis in Astrophysics\\
& Bok Center Certificate of Distinction in Teaching\\
& Wrote and delivered two class lectures\\
Jan - May 2013 & AY302: \emph{Scientists Teaching Science}, taught by Dr. Phil Sadler\\
Sep - Dec 2012 & \emph{Teaching Fellow}, AY 17: Galaxies and Cosmology\\
& Bok Center Certificate of Distinction in Teaching\\
\end{tabular*}

\section*{Professional Service and Outreach}
\begin{tabular*}{\textwidth}{@{\hspace{10pt}}p{1.4in}l}
Sep 2019 - present & Berkeley ExoCoffeeTea arXiv discussion organizer \\ 
29 Apr - 2 May, 2019 & AURA Future Leader \\
Fall 2018 & NAS Astro2020 Early Career Decadal Survey Focus Session Participant \\
2017 - 2018 & Stanford KIPAC Colloquium Committee \\
Aug 2016 & Montauk Observatory Public Lecture, Montauk, NY \\
&  \emph{East End Dark Skies Spark a Career in Astrophysics}\\
Dec 2016 & Bay Area Exoplanet Meeting LOC \\
2016 - present & Referee for the Astrophysical Journal \\
2013 - 2015 & Harvard Astronomy Department Peer mentor\\
2012 - 2013 & Harvard Undergrad Observing Project (HOP) volunteer\\
Apr 28, 2012 & Cambridge Explores the Universe, volunteer\\
Sep 2011 - Mar 2012 & Braintree High School Science Fair Mentor with students\\
& Mr. Joshua Kelleher and Mr. Brendan Newell\\
Feb 2011 - Feb 2012 & Fauquier County Light Pollution High School Science Project Mentor\\
& with student Ms. Virginia Johnson\\
Feb 8, 2012 & High Science Fair Judge, East Boston High School\\
Oct 26, 2011 & Science in the News (SITN) Public Lecture, \emph{The Chemical Enrichment of the Universe}, Boston, MA\\
Jul 2011 - 2015 & Library Committee Graduate Student Representative, Harvard-Smithsonian CfA Wolbach Library\\
Dec 2010 - 2015 & \href{http://astrobites.com/}{\emph{Astrobites}} (daily astrophysical literature journal) co-founder and contributing author\\
Oct 2009 - Apr 2010 & \emph{Dark Skies, Bright Kids} science program, rural Central Virginia\\
\end{tabular*}

\section*{Collaborative Posters}
\begin{etaremune}
\item \emph{Snapshots of the Universe: A Multi-Lingual Astronomy Art Book}\\
Beaton, Rachael; Jackson, L.; Carlberg, J.; Johnson, K.; Marchand, R.; Sivakoff, G.; \textbf{Czekala, I.}; Damke, G.; Dean, J.; Drosback, M.; Gugliucci, N.; Martinez, O.; Wong, A.; Zasowski, G.; Skies, Dark; Kids, Bright\\
American Astronomical Society, AAS Meeting \#220, \#437.13
\item \emph{Astrobites: The Astro-ph Reader's Digest For Undergraduates}\\
Sanders, Nathan; Newton, E. R.; \textbf{Czekala, I.}; Rosenfeld, K.; Dressing, C. D.; Gifford, D.; Suresh, J.; Schneider, E.; Morley, C.; Kohler, S.\\
American Astronomical Society, AAS Meeting \#218, \#333.11; Bulletin of the American Astronomical Society, Vol. 43, 2011
\end{etaremune}

\section*{References}
\begin{tabular*}{\textwidth}{@{\hspace{10pt}}p{1.9in}l}
  Professor Eugene Chiang & University of California Berkeley (echiang@astro.berkeley.edu) \\
  Professor Bruce Macintosh & Stanford University (bmacintosh@stanford.edu) \\
  Dr. Sean M. Andrews & Harvard-Smithsonian Center for Astrophysics (sandrews@cfa.harvard.edu)\\
  Professor Eric L. N. Jensen & Swarthmore College (ejensen1@swarthmore.edu) \\
  Dr. David Latham & Harvard-Smithsonian Center for Astrophysics (dlatham@cfa.harvard.edu) \\
  Professor James Moran & Harvard-Smithsonian Center for Astrophysics (jmoran@cfa.harvard.edu) \\
  Professor Kelsey Johnson & University of Virginia (kej7a@virginia.edu) \\
\end{tabular*}


\end{document}
